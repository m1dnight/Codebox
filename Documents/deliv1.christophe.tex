\documentclass[10pt,a4paper,BCOR12mm, headexclude, footexclude,
  twoside, openright]{scrartcl}
\usepackage[scaled]{helvet}
\usepackage[british]{babel}
\usepackage[utf8]{inputenc}
\usepackage[T1]{fontenc}
\usepackage{fancyhdr}
\usepackage{lastpage}
\usepackage{ifthen}
\usepackage{amsmath,amsfonts,amsthm}
\usepackage{sfmath}
\usepackage{makecell}
\usepackage{booktabs}
\usepackage{sectsty}
\usepackage{url} %url in footnote
\usepackage{color}
\usepackage{tikz}
\usepackage{graphicx}

%% Listings
\usepackage{listings}
\usepackage{xcolor}

\graphicspath{ {images/} }

%\KOMAoptions{optionenliste}
%\KOMAoptions{Option}{Werteliste}


\addtokomafont{caption}{\small}
%\setkomafont{descriptionlabel}{\normalfont
%   \bfseries}
\setkomafont{captionlabel}{\normalfont
    \bfseries}
\let\oldtabular\tabular
\renewcommand{\tabular}{\sffamily\oldtabular}
\KOMAoptions{abstract=true}
%\setkomafont{footnote}{\sffamily}
%\KOMAoptions{twoside=true}
%\KOMAoptions{headsepline=true}
%\KOMAoptions{footsepline=true}
\renewcommand\familydefault{\sfdefault}
\renewcommand{\arraystretch}{1.1}
\newcommand{\horrule}[1]{\rule{\linewidth}{#1}}
\setlength{\textheight}{230mm}
\allsectionsfont{\centering \normalfont\scshape}
\let\tmp\oddsidemargin
\let\oddsidemargin\evensidemargin
\let\evensidemargin\tmp
\reversemarginpar

\numberwithin{equation}{section} % Number equations within sections (i.e. 1.1, 1.2, 2.1, 2.2 instead of 1, 2, 3, 4)
\numberwithin{figure}{section} % Number figures within sections (i.e. 1.1, 1.2, 2.1, 2.2 instead of 1, 2, 3, 4)
\numberwithin{table}{section} % Number tables within sections (i.e. 1.1, 1.2, 2.1, 2.2 instead of 1, 2, 3, 4)

\setlength\parindent{0pt}

%%%% Listings
% C#
%\setmonofont{Consolas} %to be used with XeLaTeX or LuaLaTeX
\definecolor{bluekeywords}{rgb}{0,0,1}
\definecolor{greencomments}{rgb}{0,0.5,0}
\definecolor{redstrings}{rgb}{0.64,0.08,0.08}
\definecolor{xmlcomments}{rgb}{0.5,0.5,0.5}
\definecolor{types}{rgb}{0.17,0.57,0.68}

\usepackage{listings}
\lstset{language=[Sharp]C,
captionpos=b,
%numbers=left, %Nummerierung
%numberstyle=\tiny, % kleine Zeilennummern
frame=lines, % Oberhalb und unterhalb des Listings ist eine Linie
showspaces=false,
showtabs=false,
breaklines=true,
showstringspaces=false,
breakatwhitespace=true,
escapeinside={(*@}{@*)},
commentstyle=\color{greencomments},
morekeywords={partial, var, value, get, set},
keywordstyle=\color{bluekeywords},
stringstyle=\color{redstrings},
basicstyle=\ttfamily\small,
}

\begin{document}
%\sffamily
\fancypagestyle{plain}
{%
  \renewcommand{\headrulewidth}{0pt}%
  \renewcommand{\footrulewidth}{0.5pt}
  \fancyhf{}%
  \fancyfoot[R]{\emph{\footnotesize Page \thepage\ of \pageref{LastPage}}}%
  \fancyfoot[C]{\emph{\footnotesize Christophe De Troyer and Maarten Vandercammen}}%
}

\thispagestyle{plain}

\titlehead
{
    Vrije Universiteit Brussel\\%
    Pleinlaan 2\\%
    Software Languages Lab - Dept. Of Computer Science\hfill
    Master Studies%
}
\subject{\vspace{-1ex} \horrule{2pt}\\[0.15cm]
  {\textsc{\texttt{Capita Selecta: Software Engineering}}}}
\title{Project Software Engineering\\[0.5cm]}
\subtitle{\textsc{\texttt{Deliverable 1}}\\\horrule{2pt}\\[0.5cm]}
\author{
  \bfseries{Christophe De Troyer}\vspace{-2ex}
  \and
  \bfseries{Maarten Vandercammen}\vspace{-2ex} 
}
\date{\begin{tabular}{cc}
  \textsc{Date:}& \textsc{\emph{\today}}\\
  \textsc{Due :}& \textsc{\emph{17th June 2015}}\vspace{3ex}
\end{tabular}}
\maketitle

%\begin{abstract}
%\end{abstract}
\newpage

\tableofcontents

\newpage

%-------------------------------
\fancypagestyle{plain} {%
  \renewcommand{\headrulewidth}{0.5pt}%
  \renewcommand{\footrulewidth}{0.5pt} \fancyhf{}%
  \fancyhead[R]{\emph{\footnotesize \today}} \fancyfoot[R]{\emph{\footnotesize
      Page \thepage\ of \pageref{LastPage}}}%
  \fancyfoot[C]{\hspace*{-1.5cm} \emph{\footnotesize Christophe De Troyer}
    \hspace{2.7cm} \emph{\footnotesize Maarten Vandercammen} \\ \hspace*{-2cm}
    \emph{\footnotesize 106490} \hspace{4.5cm} \emph{\footnotesize 98341}}%
}

\pagestyle{plain}

\section{Detailed test strategy}
In order to assure proper workings of the back-end it is needed to create unit
tests that cover all of the controllers. Since this sprint will only implement
features of the high priority slice only two controllers need proper testing:
the account controller and the snippet controller.

\section{Accounts}

\paragraph{Unit Tests}

The account controller's functionality is comprised of two important operations:
logging in and registering a new account.

\begin{itemize}

\item[\textbf{Login}]To test the login process it requires some special
  testing. The implementation notifies the user of a wrong password or wrong
  email address. As such tests need to be written to assure that the system
  detects which piece of information is wrong: the username or the password and
  notifies the user as such. And finally we need to test that proper login also
  work.

\item[\textbf{Registration}] To test the registration process one has to test
  the registration process. However, this process is not contained within the
  controller but within the \texttt{Custom\-Member\-ship\-Provider}. The only
  thing to test in the \texttt{HomeController} itself was a registration with
  proper credentials.

\end{itemize}



\section{Snippets}
\paragraph{Unit Tests}

The main code unit to test for snippets was again the snippet controller. The
implemented features in this sprint were the listing of snippets and the
creation of snippets.

\begin{itemize}

\item[Listing] To test the creation of snippets one has to check the validity of the
  input at the controller. Hence, tests need to be created to edit snippets and
  to view snippets. The latter is a simple test to make sure the controller does
  a proper redirect or returns the proper snippet. The former needs to verify
  the user can only request to edit or create snippets he has access to and that
  the edits are processed properly by the controller. The storage of snippets
  can not be tested however. The application uses Entity Framework as its
  backend and thus we find that this software should not be tested.

\item[Editing] To properly unit test the listing of snippets it suffices to tes
  that the controller returns a list of all the valid snippets when a user
  requests a list.

\end{itemize}


\section{Estimates}
\subsection{Unit Tests}
The above described unit tests require the first initial setup and exploration
of the \texttt{Moq} framework and \texttt{Ninject}. As such we expect some
additional time to be spent on figuring out how everything will be done.

The unit tests should require half a day to setup and configure the sample data
and possible configuration. For the actual tests we will assume that one or two
full days should be enough to write all the tests. We predict that the testing
for the editing and listing each will be done in a single day. A day is planned
for each due to the fact that possible bugs may have to be fixed.
\subsection{Interface Tests}

\end{document}
