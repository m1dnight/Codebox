\documentclass[10pt,a4paper,BCOR12mm, headexclude, footexclude,
  twoside, openright]{scrartcl}
\usepackage[scaled]{helvet}
\usepackage[british]{babel}
\usepackage[utf8]{inputenc}
\usepackage[T1]{fontenc}
\usepackage{fancyhdr}
\usepackage{lastpage}
\usepackage{ifthen}
\usepackage{amsmath,amsfonts,amsthm}
\usepackage{sfmath}
\usepackage{makecell}
\usepackage{booktabs}
\usepackage{sectsty}
\usepackage{url} %url in footnote
\usepackage{color}
\usepackage{tikz}
\usepackage{graphicx}

%% Listings
\usepackage{listings}
\usepackage{xcolor}

\graphicspath{ {images/} }

%\KOMAoptions{optionenliste}
%\KOMAoptions{Option}{Werteliste}


\addtokomafont{caption}{\small}
%\setkomafont{descriptionlabel}{\normalfont
%   \bfseries}
\setkomafont{captionlabel}{\normalfont
    \bfseries}
\let\oldtabular\tabular
\renewcommand{\tabular}{\sffamily\oldtabular}
\KOMAoptions{abstract=true}
%\setkomafont{footnote}{\sffamily}
%\KOMAoptions{twoside=true}
%\KOMAoptions{headsepline=true}
%\KOMAoptions{footsepline=true}
\renewcommand\familydefault{\sfdefault}
\renewcommand{\arraystretch}{1.1}
\newcommand{\horrule}[1]{\rule{\linewidth}{#1}}
\setlength{\textheight}{230mm}
\allsectionsfont{\centering \normalfont\scshape}
\let\tmp\oddsidemargin
\let\oddsidemargin\evensidemargin
\let\evensidemargin\tmp
\reversemarginpar

\numberwithin{equation}{section} % Number equations within sections (i.e. 1.1, 1.2, 2.1, 2.2 instead of 1, 2, 3, 4)
\numberwithin{figure}{section} % Number figures within sections (i.e. 1.1, 1.2, 2.1, 2.2 instead of 1, 2, 3, 4)
\numberwithin{table}{section} % Number tables within sections (i.e. 1.1, 1.2, 2.1, 2.2 instead of 1, 2, 3, 4)

\setlength\parindent{0pt}

%%%% Listings
% C#
%\setmonofont{Consolas} %to be used with XeLaTeX or LuaLaTeX
\definecolor{bluekeywords}{rgb}{0,0,1}
\definecolor{greencomments}{rgb}{0,0.5,0}
\definecolor{redstrings}{rgb}{0.64,0.08,0.08}
\definecolor{xmlcomments}{rgb}{0.5,0.5,0.5}
\definecolor{types}{rgb}{0.17,0.57,0.68}

\usepackage{listings}
\lstset{language=[Sharp]C,
captionpos=b,
%numbers=left, %Nummerierung
%numberstyle=\tiny, % kleine Zeilennummern
frame=lines, % Oberhalb und unterhalb des Listings ist eine Linie
showspaces=false,
showtabs=false,
breaklines=true,
showstringspaces=false,
breakatwhitespace=true,
escapeinside={(*@}{@*)},
commentstyle=\color{greencomments},
morekeywords={partial, var, value, get, set},
keywordstyle=\color{bluekeywords},
stringstyle=\color{redstrings},
basicstyle=\ttfamily\small,
}

\begin{document}
%\sffamily
\fancypagestyle{plain}
{%
  \renewcommand{\headrulewidth}{0pt}%
  \renewcommand{\footrulewidth}{0.5pt}
  \fancyhf{}%
  \fancyfoot[R]{\emph{\footnotesize Page \thepage\ of \pageref{LastPage}}}%
  \fancyfoot[C]{\emph{\footnotesize Christophe De Troyer and Maarten Vandercammen}}%
}

\thispagestyle{plain}

\titlehead
{
    Vrije Universiteit Brussel\\%
    Pleinlaan 2\\%
    Software Languages Lab - Dept. Of Computer Science\hfill
    Master Studies%
}
\subject{\vspace{-1ex} \horrule{2pt}\\[0.15cm]
  {\textsc{\texttt{Capita Selecta: Software Engineering}}}}
\title{Project Software Engineering\\[0.5cm]}
\subtitle{\textsc{\texttt{Deliverable 1}}\\\horrule{2pt}\\[0.5cm]}
\author{
  \bfseries{Christophe De Troyer}\vspace{-2ex}
  \and
  \bfseries{Maarten Vandercammen}\vspace{-2ex} 
}
\date{\begin{tabular}{cc}
  \textsc{Date:}& \textsc{\emph{\today}}\\
  \textsc{Due :}& \textsc{\emph{17th June 2015}}\vspace{3ex}
\end{tabular}}
\maketitle

%-------------------------------
\fancypagestyle{plain}
{%
  \renewcommand{\headrulewidth}{0.5pt}%
  \renewcommand{\footrulewidth}{0.5pt}
  \fancyhf{}%
  \fancyhead[R]{\emph{\footnotesize \today}}
  \fancyfoot[R]{\emph{\footnotesize Page \thepage\ of \pageref{LastPage}}}%
  \fancyfoot[C]{\hspace*{-1.5cm} \emph{\footnotesize Christophe De Troyer} \hspace{2.7cm} \emph{\footnotesize Maarten Vandercammen} \\ \hspace*{-2cm} \emph{\footnotesize 106490} \hspace{4.5cm} \emph{\footnotesize 98341}}%
}

\pagestyle{plain}

\section{Detailed test strategy}

\subsection{Overview}
In the second iterations, we test three user-stories: editing user accounts, deleting an existing code snippet and editing an existing code snippet.
For each of these user-stories, we will create both automated GUI tests and automated back-end tests.

\subsubsection*{Back-end tests}
In order to assure proper workings of the back-end it is needed to create unit
tests that cover all of the controllers. Since this sprint will only implement
features of the medium priority slice only two controllers need proper testing:
the account controller and the snippet controller.

\subsubsection*{GUI tests}
Similar to the first iteration, all GUI tests will be performed using the Selenium framework\footnote{http://www.seleniumhq.org/}.
GUI testing will be done from a black-box perspective, so that we can model a user's experience as accurately as possible.
Specifically, this means that we will avoid inserting or updating values, such as user accounts or code snippets, directly from the database wherever possible.
Instead, we will perform these actions through the webinterface, similar to how a user would accomplish this.
However, we will insert one user account directly into the database, because most user-stories, e.g., creating, viewing or editing snippets, require the user to be logged in.
It would be undesirable to first register a user, log in with this new user account, perform the desired test and log out again, because these tests would then hinge completely on whether the 'Register' user-story succeeded.

We therefore choose to insert this user account directly into the database.

\subsection{User management}

\subsubsection*{Edit account}

\paragraph{Back-end tests}
To edit an account all that is needed really is checking the valid input and then making sure that this input is processed correctly by the repository.

\paragraph{GUI tests}
This user-story consists of editing an existing user account by changing certain attributes, such as the name, e-mail address or password. If users wish to make any changes to their password, they must verify themselves by again entering their current password.
To test this user-story, we will take two approaches. 
For tests that check invalid edit actions, e.g., not providing the correct password when necessary, entering an invalid e-mail address etc., we will log in using the user account that was inserted into the database, try to change the desired attribute and check whether the correct error message is displayed.
For tests that should succeed, we will first create an automatically generated user account, log in using this account, make the required changes and check whether they have persisted by logging out and back in, and verifying the respective attributes.
We split our tests over these two approaches because we expect that it will be undesirable to make any lasting changes to the user account that was stored directly in the database.
This account is used not only for testing this user-story, but also e.g. the 'Create snippet', 'View snippet' or 'Edit snippet' stories.
Testing this user-story might inadvertently corrupt this central user account. This would mean that the tests for our other user-stories would be unable to use this account to log in into Codebox.

As with other user-stories, we will also check invalid edits to a user account, such as entering an invalid password, entering an invalid new password or an incorrect current password, leaving all textboxes blank etc.

\subsection{Snippet management}

\subsubsection*{Delete snippet}
\paragraph{Back-end tests}
To delete a snippet three things need to be in order: the deleter is allowed to delete, the program fetches delete instruction on non-existing identifiers and the delete operation actually deletes the snippet.
All this can be tested by writing unit tests against the snippet controller.

\paragraph{GUI tests}
Testing the deletion of a snippet can be accomplished by creating a new, automatically generated snippet, navigating to the snippet overview page and simulate a mouseclick on the delete-link of this snippet.
If successful, a message should appear that the snippet has been deleted and the snippet should be removed from the overview.
All of this can be accomplished with the use of the Selenium framework: we only have to make sure that the correct DOM-element appears, and that the link to the deleted snippet is removed from the overview.

\subsubsection*{Edit snippet}

\paragraph{Back-end tests}

\paragraph{GUI tests}
Creating a GUI test that checks if snippets are correctly edited can be accomplished by automatically generating and creating new snippets, clicking on the link to the edit-page for the newly created snippet, altering the desired attributes and clicking the 'Save' button. We will then check whether the attributes were correctly updated by navigating back to the webpage of this snippet and checking on the value of the attributes that were altered.
The webform that is used to edit snippets is identical to the form used to create snippets, so we expect that most code for filling in these forms can be reused from the GUI tests for the 'Create snippet' user-story.

A snippet can take a completely arbitrary value for any of its attributes, but its name and code fragment must be non-empty.
Therefore, similar to the 'Create new snippet' user-story, the only way of incorrectly editing an existing snippet is by erasing its name or code fragment and attempting to save these changes.
We will check this scenario by doing exactly this and verifying whether the correct error messages appear.


\section{Estimates}
\subsection{Unit Tests}
For the unit tests two days will be planned in again. To make sure there is time
to fix any possible bugs while writing the unit tests. Hence, day one will be
used to implement the editing of accounts and the second day will be used to
implement the unit tests for the deletion of snippets.
\subsection{Interface Tests}

\section{Evaluation of previous planning}
The initial project setup was rather difficult. Since this project is a group
effor we had to synchronize on how to setup the source repository. The first
time consuming task was getting Maarten's laptop setup properly. His laptop did
not have any of the required software installed. The second time consuming issue
was the configuration of the repository. Neither of the group members had
extensive experience with version controlling .net projects and as such we ran
into some issues with the dependency management of packages. After this initial
hurdle was overcome we could start on our own individual tasks.

\subsection{Unit Tests}
The unit tests went according to plan. There was however one issue that had to
be resolved. To unit test the \texttt{FormsAuthentication} is a static
class. The class serves the purpose of managing authentication on the
website. As such it was impossible to use \texttt{Moq} to override the
behaviour. To mitigate this issue it suffices to wrap the class in a non-static
version and inject it using \texttt{Ninject}. This little hurdle was resolved
rather quickly by means of some Google-Fu.

\end{document}