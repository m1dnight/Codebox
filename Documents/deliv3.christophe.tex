\documentclass[10pt,a4paper,BCOR12mm, headexclude, footexclude,
  twoside, openright]{scrartcl}
\usepackage[scaled]{helvet}
\usepackage[british]{babel}
\usepackage[utf8]{inputenc}
\usepackage[T1]{fontenc}
\usepackage{fancyhdr}
\usepackage{lastpage}
\usepackage{ifthen}
\usepackage{amsmath,amsfonts,amsthm}
\usepackage{sfmath}
\usepackage{makecell}
\usepackage{booktabs}
\usepackage{sectsty}
\usepackage{url} %url in footnote
\usepackage{color}
\usepackage{tikz}
\usepackage{graphicx}

%% Listings
\usepackage{listings}
\usepackage{xcolor}

\graphicspath{ {images/} }

%\KOMAoptions{optionenliste}
%\KOMAoptions{Option}{Werteliste}


\addtokomafont{caption}{\small}
%\setkomafont{descriptionlabel}{\normalfont
%   \bfseries}
\setkomafont{captionlabel}{\normalfont
    \bfseries}
\let\oldtabular\tabular
\renewcommand{\tabular}{\sffamily\oldtabular}
\KOMAoptions{abstract=true}
%\setkomafont{footnote}{\sffamily}
%\KOMAoptions{twoside=true}
%\KOMAoptions{headsepline=true}
%\KOMAoptions{footsepline=true}
\renewcommand\familydefault{\sfdefault}
\renewcommand{\arraystretch}{1.1}
\newcommand{\horrule}[1]{\rule{\linewidth}{#1}}
\setlength{\textheight}{230mm}
\allsectionsfont{\centering \normalfont\scshape}
\let\tmp\oddsidemargin
\let\oddsidemargin\evensidemargin
\let\evensidemargin\tmp
\reversemarginpar

\numberwithin{equation}{section} % Number equations within sections (i.e. 1.1, 1.2, 2.1, 2.2 instead of 1, 2, 3, 4)
\numberwithin{figure}{section} % Number figures within sections (i.e. 1.1, 1.2, 2.1, 2.2 instead of 1, 2, 3, 4)
\numberwithin{table}{section} % Number tables within sections (i.e. 1.1, 1.2, 2.1, 2.2 instead of 1, 2, 3, 4)

\setlength\parindent{0pt}

%%%% Listings
% C#
%\setmonofont{Consolas} %to be used with XeLaTeX or LuaLaTeX
\definecolor{bluekeywords}{rgb}{0,0,1}
\definecolor{greencomments}{rgb}{0,0.5,0}
\definecolor{redstrings}{rgb}{0.64,0.08,0.08}
\definecolor{xmlcomments}{rgb}{0.5,0.5,0.5}
\definecolor{types}{rgb}{0.17,0.57,0.68}

\usepackage{listings}
\lstset{language=[Sharp]C,
captionpos=b,
%numbers=left, %Nummerierung
%numberstyle=\tiny, % kleine Zeilennummern
frame=lines, % Oberhalb und unterhalb des Listings ist eine Linie
showspaces=false,
showtabs=false,
breaklines=true,
showstringspaces=false,
breakatwhitespace=true,
escapeinside={(*@}{@*)},
commentstyle=\color{greencomments},
morekeywords={partial, var, value, get, set},
keywordstyle=\color{bluekeywords},
stringstyle=\color{redstrings},
basicstyle=\ttfamily\small,
}

\begin{document}
%\sffamily
\fancypagestyle{plain}
{%
  \renewcommand{\headrulewidth}{0pt}%
  \renewcommand{\footrulewidth}{0.5pt}
  \fancyhf{}%
  \fancyfoot[R]{\emph{\footnotesize Page \thepage\ of \pageref{LastPage}}}%
  \fancyfoot[C]{\emph{\footnotesize Christophe De Troyer and Maarten Vandercammen}}%
}

\thispagestyle{plain}

\titlehead
{
    Vrije Universiteit Brussel\\%
    Pleinlaan 2\\%
    Software Languages Lab - Dept. Of Computer Science\hfill
    Master Studies%
}
\subject{\vspace{-1ex} \horrule{2pt}\\[0.15cm]
  {\textsc{\texttt{Capita Selecta: Software Engineering}}}}
\title{Project Software Engineering\\[0.5cm]}
\subtitle{\textsc{\texttt{Deliverable 2}}\\\horrule{2pt}\\[0.5cm]}
\author{
  \bfseries{Christophe De Troyer}\vspace{-2ex}
  \and
  \bfseries{Maarten Vandercammen}\vspace{-2ex} 
}
\date{\begin{tabular}{cc}
  \textsc{Date:}& \textsc{\emph{\today}}\\
  \textsc{Due :}& \textsc{\emph{17th June 2015}}\vspace{3ex}
\end{tabular}}
\maketitle

%\begin{abstract}
%\end{abstract}
\newpage

\tableofcontents

\newpage

%-------------------------------
\fancypagestyle{plain} {%
  \renewcommand{\headrulewidth}{0.5pt}%
  \renewcommand{\footrulewidth}{0.5pt} \fancyhf{}%
  \fancyhead[R]{\emph{\footnotesize \today}} \fancyfoot[R]{\emph{\footnotesize
      Page \thepage\ of \pageref{LastPage}}}%
  \fancyfoot[C]{\hspace*{-1.5cm} \emph{\footnotesize Christophe De Troyer}
    \hspace{2.7cm} \emph{\footnotesize Maarten Vandercammen} \\ \hspace*{-2cm}
    \emph{\footnotesize 106490} \hspace{4.5cm} \emph{\footnotesize 98341}}%
}

\pagestyle{plain}

\section{Detailed test strategy}
In order to assure proper workings of the back-end it is needed to create unit
tests that cover all of the controllers. Since this sprint will only implement
features of the low priority slice only a single controller needs testing: the
group controller.



\section{Groups}
\paragraph{Unit Tests}
To make sure the group controller acts as it is supposed to it has to be tested
on each function. The group controller is used to create, edit and delete
groups. As such, to achieve optimal coverage we will test all of these
functions. 

The main strategy for testing is in line with the strategy that was applied to
the other controllers. Test creating with correct data and faulty data. Deletion
will be tested to make sure the user has appropriate authorization to
remove. Ensure that the deletion is actually performed and check that
modifications are actually executed.

\section{Evaluation of previous planning}
The planned tasks were executed within the set timeslot. As such we can safely
state that the previous planning was correct.
\subsection{Unit Tests}
The planned tasks have been executed in the time we planned for them in the
previous deliverable. Again there was a small hurdle in the implementation of
the project that required refactoring. This time the
\texttt{CustomMembershiProvider} implemented \texttt{protected} methods. These
methods were thus not accessible from within the unit tests. As such a derived
class was created from the \texttt{CustomMembershipProvider} which publishes the
methods in a public fashion. Except that hurdle the tests were straightforward
as they were a variation on the previously written tests.
\end{document}
