\documentclass[10pt,a4paper,BCOR12mm, headexclude, footexclude,
  twoside, openright]{scrartcl}
\usepackage[scaled]{helvet}
\usepackage[british]{babel}
\usepackage[utf8]{inputenc}
\usepackage[T1]{fontenc}
\usepackage{fancyhdr}
\usepackage{lastpage}
\usepackage{ifthen}
\usepackage{amsmath,amsfonts,amsthm}
\usepackage{sfmath}
\usepackage{makecell}
\usepackage{booktabs}
\usepackage{sectsty}
\usepackage{url} %url in footnote
\usepackage{color}
\usepackage{tikz}
\usepackage{graphicx}

%% Listings
\usepackage{listings}
\usepackage{xcolor}

\graphicspath{ {images/} }

%\KOMAoptions{optionenliste}
%\KOMAoptions{Option}{Werteliste}


\addtokomafont{caption}{\small}
%\setkomafont{descriptionlabel}{\normalfont
%   \bfseries}
\setkomafont{captionlabel}{\normalfont
    \bfseries}
\let\oldtabular\tabular
\renewcommand{\tabular}{\sffamily\oldtabular}
\KOMAoptions{abstract=true}
%\setkomafont{footnote}{\sffamily}
%\KOMAoptions{twoside=true}
%\KOMAoptions{headsepline=true}
%\KOMAoptions{footsepline=true}
\renewcommand\familydefault{\sfdefault}
\renewcommand{\arraystretch}{1.1}
\newcommand{\horrule}[1]{\rule{\linewidth}{#1}}
\setlength{\textheight}{230mm}
\allsectionsfont{\centering \normalfont\scshape}
\let\tmp\oddsidemargin
\let\oddsidemargin\evensidemargin
\let\evensidemargin\tmp
\reversemarginpar

\numberwithin{equation}{section} % Number equations within sections (i.e. 1.1, 1.2, 2.1, 2.2 instead of 1, 2, 3, 4)
\numberwithin{figure}{section} % Number figures within sections (i.e. 1.1, 1.2, 2.1, 2.2 instead of 1, 2, 3, 4)
\numberwithin{table}{section} % Number tables within sections (i.e. 1.1, 1.2, 2.1, 2.2 instead of 1, 2, 3, 4)

\setlength\parindent{0pt}

%%%% Listings
% C#
%\setmonofont{Consolas} %to be used with XeLaTeX or LuaLaTeX
\definecolor{bluekeywords}{rgb}{0,0,1}
\definecolor{greencomments}{rgb}{0,0.5,0}
\definecolor{redstrings}{rgb}{0.64,0.08,0.08}
\definecolor{xmlcomments}{rgb}{0.5,0.5,0.5}
\definecolor{types}{rgb}{0.17,0.57,0.68}

\usepackage{listings}
\lstset{language=[Sharp]C,
captionpos=b,
%numbers=left, %Nummerierung
%numberstyle=\tiny, % kleine Zeilennummern
frame=lines, % Oberhalb und unterhalb des Listings ist eine Linie
showspaces=false,
showtabs=false,
breaklines=true,
showstringspaces=false,
breakatwhitespace=true,
escapeinside={(*@}{@*)},
commentstyle=\color{greencomments},
morekeywords={partial, var, value, get, set},
keywordstyle=\color{bluekeywords},
stringstyle=\color{redstrings},
basicstyle=\ttfamily\small,
}

\begin{document}
%\sffamily
\fancypagestyle{plain}
{%
  \renewcommand{\headrulewidth}{0pt}%
  \renewcommand{\footrulewidth}{0.5pt}
  \fancyhf{}%
  \fancyfoot[R]{\emph{\footnotesize Page \thepage\ of \pageref{LastPage}}}%
  \fancyfoot[C]{\emph{\footnotesize Christophe De Troyer and Maarten Vandercammen}}%
}

\thispagestyle{plain}

\titlehead
{
    Vrije Universiteit Brussel\\%
    Pleinlaan 2\\%
    Software Languages Lab - Dept. Of Computer Science\hfill
    Master Studies%
}
\subject{\vspace{-1ex} \horrule{2pt}\\[0.15cm]
  {\textsc{\texttt{Capita Selecta: Software Engineering}}}}
\title{Project Software Engineering\\[0.5cm]}
\subtitle{\textsc{\texttt{Deliverable 1}}\\\horrule{2pt}\\[0.5cm]}
\author{
  \bfseries{Christophe De Troyer}\vspace{-2ex}
  \and
  \bfseries{Maarten Vandercammen}\vspace{-2ex} 
}
\date{\begin{tabular}{cc}
  \textsc{Date:}& \textsc{\emph{\today}}\\
  \textsc{Due :}& \textsc{\emph{17th June 2015}}\vspace{3ex}
\end{tabular}}
\maketitle

%\begin{abstract}
%\end{abstract}
\newpage

\tableofcontents

\newpage

%-------------------------------
\fancypagestyle{plain}
{%
  \renewcommand{\headrulewidth}{0.5pt}%
  \renewcommand{\footrulewidth}{0.5pt}
  \fancyhf{}%
  \fancyhead[R]{\emph{\footnotesize \today}}
  \fancyfoot[R]{\emph{\footnotesize Page \thepage\ of \pageref{LastPage}}}%
  \fancyfoot[C]{\hspace*{-1.5cm} \emph{\footnotesize Christophe De Troyer} \hspace{2.7cm} \emph{\footnotesize Maarten Vandercammen} \\ \hspace*{-2cm} \emph{\footnotesize 106490} \hspace{4.5cm} \emph{\footnotesize 98341}}%
}

\pagestyle{plain}

\section{Detailed test strategy}

Unit-tests for the GUI of each user story will be created with the use of the Selenium framework \footnote{http://www.seleniumhq.org/}.
This framework can be used to simulate a user's interaction with the website. Specifically, it offers mechanisms to send mouse-clicks to DOM-elements, navigate a website, or send keyboard input through input forms.
Conversely, it also offers mechanisms to check the complete state of a DOM-element.
Creating automated GUI tests can then be accomplished by modelling the input that can be send by user and checking whether the DOM-elements which are expected to change have been correctly manipulated.

GUI tests consist of both valid actions performed by the user, such as successfully creating a new user account, logging in or creating new snippets, and invalid actions, such as logging in unsuccessfully, creating invalid snippets, etc.

In this project, GUI tests will be written from a black-box perspective. Although we aim to make the GUI tests fine-grained, we will not check the exact errors raised by the application. We only check how these errors are presented to the user.

\subsubsection*{Log in}
The result of the action of a user logging in can be categorized in three main scenario's: either the log-in was successful, either it was unsuccessful because a non-existing username was provided, or it was unsuccessful because an incorrect password was entered.
We will test these three scenario's with the use of Selenium.
We navigate to the home page of Codebox, navigate from there to the log-in page, find the DOM-elements that correspond with the textboxes where users enter their username and password, enter some credentials and subsequently click the 'Log in' button.
A successful log-in can be validated by checking whether the system redirected to the correct webpage and is showing the correct credentials.
An invalid log-in can be tested by checking whether the expected error message appears on the screen.

The first scenario requires that at least one user account has already been created in Codebox, so that a log-in can be performed using this account.
For this reason, we will insert one or multiple user profiles directly into the database of Codebox and check whether it is possible to log in using the username and password of the respective users.
By inserting these profiles directly into the database, we can test the 'Log in' user-story separately from e.g., the 'Register' user-story.

\subsubsection*{Register}

\subsubsection*{Log out}

\subsubsection*{Create new snippet}

\subsubsection*{Show snippet}

\end{document}
