\documentclass[10pt,a4paper,BCOR12mm, headexclude, footexclude,
  twoside, openright]{scrartcl}
\usepackage[scaled]{helvet}
\usepackage[british]{babel}
\usepackage[utf8]{inputenc}
\usepackage[T1]{fontenc}
\usepackage{fancyhdr}
\usepackage{lastpage}
\usepackage{ifthen}
\usepackage{amsmath,amsfonts,amsthm}
\usepackage{sfmath}
\usepackage{makecell}
\usepackage{booktabs}
\usepackage{sectsty}
\usepackage{url} %url in footnote
\usepackage{color}
\usepackage{tikz}
\usepackage{graphicx}

%% Listings
\usepackage{listings}
\usepackage{xcolor}

\graphicspath{ {images/} }

%\KOMAoptions{optionenliste}
%\KOMAoptions{Option}{Werteliste}


\addtokomafont{caption}{\small}
%\setkomafont{descriptionlabel}{\normalfont
%   \bfseries}
\setkomafont{captionlabel}{\normalfont
    \bfseries}
\let\oldtabular\tabular
\renewcommand{\tabular}{\sffamily\oldtabular}
\KOMAoptions{abstract=true}
%\setkomafont{footnote}{\sffamily}
%\KOMAoptions{twoside=true}
%\KOMAoptions{headsepline=true}
%\KOMAoptions{footsepline=true}
\renewcommand\familydefault{\sfdefault}
\renewcommand{\arraystretch}{1.1}
\newcommand{\horrule}[1]{\rule{\linewidth}{#1}}
\setlength{\textheight}{230mm}
\allsectionsfont{\centering \normalfont\scshape}
\let\tmp\oddsidemargin
\let\oddsidemargin\evensidemargin
\let\evensidemargin\tmp
\reversemarginpar

\numberwithin{equation}{section} % Number equations within sections (i.e. 1.1, 1.2, 2.1, 2.2 instead of 1, 2, 3, 4)
\numberwithin{figure}{section} % Number figures within sections (i.e. 1.1, 1.2, 2.1, 2.2 instead of 1, 2, 3, 4)
\numberwithin{table}{section} % Number tables within sections (i.e. 1.1, 1.2, 2.1, 2.2 instead of 1, 2, 3, 4)

\setlength\parindent{0pt}

%%%% Listings
% C#
%\setmonofont{Consolas} %to be used with XeLaTeX or LuaLaTeX
\definecolor{bluekeywords}{rgb}{0,0,1}
\definecolor{greencomments}{rgb}{0,0.5,0}
\definecolor{redstrings}{rgb}{0.64,0.08,0.08}
\definecolor{xmlcomments}{rgb}{0.5,0.5,0.5}
\definecolor{types}{rgb}{0.17,0.57,0.68}

\usepackage{listings}
\lstset{language=[Sharp]C,
captionpos=b,
%numbers=left, %Nummerierung
%numberstyle=\tiny, % kleine Zeilennummern
frame=lines, % Oberhalb und unterhalb des Listings ist eine Linie
showspaces=false,
showtabs=false,
breaklines=true,
showstringspaces=false,
breakatwhitespace=true,
escapeinside={(*@}{@*)},
commentstyle=\color{greencomments},
morekeywords={partial, var, value, get, set},
keywordstyle=\color{bluekeywords},
stringstyle=\color{redstrings},
basicstyle=\ttfamily\small,
}

\begin{document}
%\sffamily
\fancypagestyle{plain}
{%
  \renewcommand{\headrulewidth}{0pt}%
  \renewcommand{\footrulewidth}{0.5pt}
  \fancyhf{}%
  \fancyfoot[R]{\emph{\footnotesize Page \thepage\ of \pageref{LastPage}}}%
  \fancyfoot[C]{\emph{\footnotesize Christophe De Troyer and Maarten Vandercammen}}%
}

\thispagestyle{plain}

\titlehead
{
    Vrije Universiteit Brussel\\%
    Pleinlaan 2\\%
    Software Languages Lab - Dept. Of Computer Science\hfill
    Master Studies%
}
\subject{\vspace{-1ex} \horrule{2pt}\\[0.15cm]
  {\textsc{\texttt{Capita Selecta: Software Engineering}}}}
\title{Project Software Engineering\\[0.5cm]}
\subtitle{\textsc{\texttt{Deliverable 3}}\\\horrule{2pt}\\[0.5cm]}
\author{
  \bfseries{Christophe De Troyer}\vspace{-2ex}
  \and
  \bfseries{Maarten Vandercammen}\vspace{-2ex} 
}
\date{\begin{tabular}{cc}
  \textsc{Date:}& \textsc{\emph{\today}}\\
  \textsc{Due :}& \textsc{\emph{17th June 2015}}\vspace{3ex}
\end{tabular}}
\maketitle

%-------------------------------
\fancypagestyle{plain}
{%
  \renewcommand{\headrulewidth}{0.5pt}%
  \renewcommand{\footrulewidth}{0.5pt}
  \fancyhf{}%
  \fancyhead[R]{\emph{\footnotesize \today}}
  \fancyfoot[R]{\emph{\footnotesize Page \thepage\ of \pageref{LastPage}}}%
  \fancyfoot[C]{\hspace*{-1.5cm} \emph{\footnotesize Christophe De Troyer} \hspace{2.7cm} \emph{\footnotesize Maarten Vandercammen} \\ \hspace*{-2cm} \emph{\footnotesize 106490} \hspace{4.5cm} \emph{\footnotesize 98341}}%
}

\pagestyle{plain}

\section{Evaluation of previous planning}

\subsection{Back-end tests}
Writing the tests for the Group controller was more straight forward than
anticipated. The controller allowed itself for easy testing and we did not hit
any hurdles like we did with the other controllers. As such we finish in
time. However, there are some tests that have suffered under the time pressure
of our theses. As such tests for editing and deleting a group are not
implemented. However, we plan on doing them afterwards. 

\subsection{GUI tests}
As expected, writing the GUI tests for the third iteration was complicated
because of the fact that we had to spend a lot of time finishing up our master's
theses.

We had planned on writing all GUI tests for the third iteration in three
days. We were able to accomplish this, but we had to drop one aspect of our GUI
tests for the 'Upload profile picture' user-story.

Originally, we planned on verifying whether users could correctly upload an
image by letting our tests upload a fixed image, calculate the hashsum of the
image that became visible, and compare this with the known hashsum of that
image.  However, implementing this was very tricky and we were plagued by
unexpected errors that we could not manage to resolve.

Because we could not spend any more time on developing this test, due to our
theses, we made the decision to drop this part of the test.  Instead, we now
verify this user-story in a much less intricate, but also less failproof, way:
we check whether a web-element is created at the expected location, and whether
all attributes of this element, i.e., its type, width, height etc., are correct.

Although this does not provide any solid guarantees that the user-story is
implemented correctly, it should offer some basic guarantees.

\end{document}